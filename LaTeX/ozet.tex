Denizli’de bulunan Öğrenci İşleri kafesine ait hem bir web uygulama projesi hem de bir mobil uygulama projesidir.

Öğrenci İşleri uygulaması: Kafede gerçekleşen Bilgi yarışmaları, Tiyatro gösterileri, Film geceleri, Tanışma kahvaltıları gibi daha birçok sosyal etkinlikler oluşturabilmemize olanak sağlar ve etkinliklere katılma başvurusu yapabileceğimiz bir mobil uygulamadır. Bu uygulama aynı zamanda müşterilerin QR kod ile menüyü görebileceği, sipariş verebileceği ve kasaya gitmeden hesap ödeyebileceği bir uygulamadır. 

Öğrenci İşleri uygulamasının amaçlarını düşünecek olursak eğer 1. Amaç: Menüde değişiklik olduğu zaman kafe sahibini menü basım maliyetinden kurtarmaya çalışmak. 2. Amaç: Kafe sahibini fazla iş gücünden kurtarmaya çalışmaktır.

Öğrenci İşleri uygulamasının önemini düşünecek olursak eğer 1. Önem: Etkinlikler ile alakalı olan işlemlerin uygulama üzerinden yapılması sayesinde minimum iş gücü ile maksimum verim elde edilmiş olunacak. 2. Önem: Müşteri QR kod sayesinde menüyü görebileceği için minimum maliyet ile maksimum kâr elde edilmiş olunacak. 3. Önem: Müşteri kasanın önünde vakit harcamaktan kurtulmuş olacaktır.

Öğrenci İşleri projesinin diğer projelerden farkına değinecek olursak eğer, 1. fark: diğer uygulamalar, sadece mobil bir uygulamadır. Öğrenci İşleri hem mobil bir uygulama hem de bir web uygulamasıdır. 2. fark: diğer uygulamalar da sadece masalar ve menü ile ilgili işlemler yapılmaktadır. Öğrenci İşlerinde ise etkinlikleri görme, yeni bir etkinlik oluşturma, etkinliklere katılım yapma gibi daha birçok işlem yapmaya olanak sağlar.

Öğrenci İşleri uygulamasını geliştirirken kullanacak olduğum teknolojiler ise şu şekildedir: Projenin Mobil uygulamasındaki frontend alanını "React Native", Web uygulamasındaki frontend alanını ise "React" ile geliştireceğim. Projenin hem mobil hem de web uygulamasındaki backend alanını ise "Firebase" ile geliştireceğim. Bu teknolojileri seçmemde ki en önemli neden günümüzde çok popüler oldukları için bu teknolojilerin kaynakları fazladır. 

Projenin veritabanı kısmında ise şu veriler bulunacaktır: Kullanıcıya ait bilgiler. (Ad – Soyad, Telefon numarası, Email adresi...) / Kullanıcı etkinlik mi oluşturmak istiyor yoksa etkinliğe katılmak mı istiyor? / Kullanıcı nasıl bir etkinlik düzenlemek istiyor? (Toplantı, Kahvaltı, Oyun...) / Kullanıcı hangi etkinliğe geldi? (Bilgi yarışması, Tiyatro gösterisi…) / Grup olarak mı geldiler yoksa tek başına mı geldi? / Gruba ait bilgiler. (Grup adı, Kişi sayısı…) / Menüye ait veriler. (Ürünün kategorisi, Ürünün adı, Ürünün fiyatı…)

\textbf{Anahtar Kelimeler:} Mobil uygulama, Web uygulaması, Sosyal etkinlikler, Başvuru, Menü, QR kod, Maksimum kâr, Minimum iş, React Native, React, Firebase, Veriler.
