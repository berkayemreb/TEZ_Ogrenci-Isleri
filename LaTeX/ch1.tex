%%%%%%%%%%%%%%%%%%%%%%%%%%%%%%%%%%%%%%%%%%%%%%%%%%%%%%%%%%%%%%%%%
\chapter{GİRİŞ - ÖĞRENCİ İŞLERİ NEDİR?}\label{giris}
\section{Öğrenci İşleri}\label{tezintanimi}
%%%%%%%%%%%%%%%%%%%%%%%%%%%%%%%%%%%%%%%%%%%%%%%%%%%%%%%%%%%%%%%%%
Denizli’de bulunan Öğrenci İşleri kafesine ait hem mobil bir uygulama hem de bir web uygulamasıdır. Kafede gerçekleşen bilgi yarışmaları, tiyatro gösterileri, film geceleri, tanışma kahvaltıları gibi daha birçok sosyal etkinlik oluşturabilmemize olanak sağlar. Aynı zamanda etkinliklere katılma başvurusu da yapabiliriz. Ayrıca müşterilerin QR kod ile menüyü görebileceği bir uygulamadır.

\subsection{Öğrenci işlerinin amacı}\label{tezinamaci}

Menüde değişiklik olduğu zaman kafe sahibini menü basım maliyetinden kurtarmaya çalışmak ve kafe sahibini fazla iş gücünden kurtarmaya çalışmaktır.

\subsection{Öğrenci işlerinin önemi}\label{tezinönemi}

Etkinlikler ile alakalı olan işlemlerin uygulama üzerinden yapılması sayesinde minimum iş gücü ile maksimum verim elde edilmiş olunacaktır. Ayrıca Müşteri QR kod sayesinde menüyü görebileceği için minimum maliyet ile maksimum kâr elde edilmiş olunacaktır.


\subsection{Öğrenci işlerinin farkı}\label{tezinfarkı}

Diğer uygulamalar sadece mobil bir uygulamadır fakat Öğrenci İşleri ise hem mobil bir uygulama hem de bir web uygulamasıdır. Ayrıca diğer uygulamalarda sadece masalar ve menü ile ilgili işlemler yapılmaktadır fakat Öğrenci İşleri ise etkinlikleri görme, yeni bir etkinlik oluşturma, etkinliklere katılım yapma gibi daha birçok işlem daha yapmaya olanak sağlar.

\section{Hipotez}\label{hipotez}

Öğrenci İşleri uygulaması kafe sahibini olabildiğince minimum maliyet ve minimum iş gücü ile maksimum verim elde etmesine olanak sağlamayı hedeflemektedir.